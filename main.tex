\documentclass[letterpaper]{twentysecondcv} % a4paper for A4

%----------------------------------------------------------------------------------------
%	 PERSONAL INFORMATION
%----------------------------------------------------------------------------------------

% If you don't need one or more of the below, just remove the content leaving the command, e.g. \cvnumberphone{}

\profilepic{resources/profile.jpeg} % Profile picture

\cvname{Marco Maida} % Your name
\cvjobtitle{Computer scientist} % Job title/career

\cvdate{13 August 1994} % Date of birth
\cvaddress{} % Short address/location, use \newline if more than 1 line is required
\cvnumberphone{} % Phone number
\cvsite{www.maida.me} % Personal website

% Avoid scrapers
\def\maila{mma}
\def\mailb{idacs}
\def\mailc{@gma}
\def\maild{il.com}
\cvmail{\maila\mailb\mailc\maild} % Email address

%----------------------------------------------------------------------------------------

\begin{document}


%----------------------------------------------------------------------------------------

\makeprofile

I am comfortable working on large code bases and designing complex systems. I have extensive experience with \textbf{C++}, \textbf{Rust}, \textbf{C}, \textbf{Python}, \textbf{C\#}, and \textbf{Coq} in production enviroments. I am \textbf{outgoing} and I \textbf{love working in teams}.

\vspace{.1cm}

\section{Software Engineering}

\begin{twenty} % Environment for a list with descriptions
	\twentyitem{Since 2022}{Software engineer -- Robotics and AI.}{Wayve Ltd}{ 
		I designed, implemented, and maintained several key systems of the on-board software stack of Wayve's autonomous cars.
		The software collects frames from different sensors; aligns, packages, and uploads enormous amounts of data, and runs a neural network for real-time inference. \textit{(C++, Rust, Python)}. }
	\twentyitem{2016-2019}{Software Engineer -- Videogames.}{34BigThings}{ 
		I contributed to five major titles, designing and implementing game infrastructure, AIs, gameplays, and dev tools. I worked on single-player and online-multiplayer games shipped on Steam, PS4, XboxOne, Switch, and mobiles. \textit{(Unity3D, C\# and Unreal Engine, C++)}}
	\twentyitem{2015-2016}{Software Engineer -- Videogames, Simulation}{Maserati, Teoresi, Choralia}{ This experience consists of two freelance B2B projects. First, I built an educational game for mobiles and browsers. I led the project end-to-end, managing one artist I hired. Second, I collaborated with an engineer and an artist in creating a 3D visualization tool used for product presentation. \textit{(C\#, Javascript)} }
	\twentyitem{2013-2016}{Software Engineer -- Industrial Software.}{R.O. srl}{
		I designed and maintained a suite of software solutions for glass processing factories. The software tracks orders and minimises the cutting machines' work, product waste, and logistic delays. I started as an individual contributor and later transitioned to managing a team of 4 engineers \textit{(C, C++, C\#, SQL)} }
\end{twenty}

\section{Education and Research}

\begin{twenty} % Environment for a list with descriptions
	\twentyitem{2022}{Research Internship}{Bloomberg LP}{ I worked on accelerating SAT solving using GPUs \textit{(C++, CUDA)}.  }
	\twentyitem{2019-2022}{PhD Student.}{Max Planck Institute}{ I worked at the intersection of formal verification and real-time systems. Additionally, I worked on trace-based schedulability analysis on Linux systems. I mentored three interns and published three papers. \textit{(COQ, C, Rust)}}
	\twentyitem{2019-2022}{Master in Computer Science.}{Technische Universität Kaiserslautern}{ I specialised in real-time systems.}
	\twentyitem{2016-2019}{Bachelor in Computer Science.}{Università degli studi di Torino}{ I specialised in computability and formal methods.}
\end{twenty}
Publications

\begin{twenty} % Environment for a list with descriptions
	\twentyitem{2021}{Foundational Response-Time Analysis as
	Explainable Evidence of Timeliness.}{Max Planck institute}{ 
		I developed POET, a tool that yields a formally verified worst-case-scenario timing analysis. I first-authored a publication at ECRTS2022, winning its \textit{outstanding paper} award.\\
		\scriptsize (\url{https://drops.dagstuhl.de/opus/volltexte/2022/16336/pdf/LIPIcs-ECRTS-2022-19.pdf})\normalfont 
	}
	\twentyitem{2022}{From Intuition to Coq: A Case Study in Verified Response-Time Analysis, FIFO Scheduling.}{Max Planck institute}{ 
		We developed a formally verified response-time analysis for FIFO schedulers, challenging the classic pen-and-paper 
		% \mbox prevents ugly line break
		\mbox{approach}. \scriptsize (\url{https://people.mpi-sws.org/~kbedarka/rtss22.pdf})\normalfont 
	}
\end{twenty}

\section{Private \& Open-Source Projects}

\begin{twenty} % Environment for a list with descriptions
	
	\twentyitem{2022}{Treecode.}{Personal}{
		I created a novel 2D scannable code that encodes messages as unique trees. 	
	 	\scriptsize (\url{www.maida.me/treecode})\normalfont
	 }
	\twentyitem{2021}{POET.}{Max Planck Institute}{
		The tool I created as part of my first academic publication. It is the first-ever implementation of a foundational response-time analysis.
	 	\scriptsize (\url{https://gitlab.mpi-sws.org/RT-PROOFS/POET})\normalfont
	 }
	 \twentyitem{2020-2022}{PROSA.}{Max Planck Institute}{
		I was one of the main contributors of PROSA during my stay at MPI-SWS. PROSA is one of the most influential formally-verified frameworks in the real-time systems' community. 
	 	\scriptsize (\url{https://gitlab.mpi-sws.org/RT-PROOFS/rt-proofs})\normalfont
	 }
	\twentyitem{2018}{Fast Mobile Cycle (FMC) Framework and Toolkit.}{34BigThings}{ 
		I developed a Unity3D framework that accelerates the creation of production-ready casual games, paired by a Python toolkit to execute bulk operations on the games.
		%With FMC, I, an artist, and a designer developed ten production-ready games within a month. 
		\scriptsize (\url{www.github.com/34OpenThings})\normalfont
	 }
\end{twenty}

%
%----------------------------------------------------------------------------------------
%	 SECOND PAGE EXAMPLE
%----------------------------------------------------------------------------------------
%
% \newpage % Start a new page

% \makeprofile % Print the sidebar
%
%\section{Other information}
%
%\subsection{Review}
%
%Alice approaches Wonderland as an anthropologist, but maintains a strong sense of noblesse oblige that comes with her class status. She has confidence in her social position, education, and the Victorian virtue of good manners. Alice has a feeling of entitlement, particularly when comparing herself to Mabel, whom she declares has a ``poky little house," and no toys. Additionally, she flaunts her limited information base with anyone who will listen and becomes increasingly obsessed with the importance of good manners as she deals with the rude creatures of Wonderland. Alice maintains a superior attitude and behaves with solicitous indulgence toward those she believes are less privileged.
%
%\section{Other information}
%
%\subsection{Review}
%
%Alice approaches Wonderland as an anthropologist, but maintains a strong sense of noblesse oblige that comes with her class status. She has confidence in her social position, education, and the Victorian virtue of good manners. Alice has a feeling of entitlement, particularly when comparing herself to Mabel, whom she declares has a ``poky little house," and no toys. Additionally, she flaunts her limited information base with anyone who will listen and becomes increasingly obsessed with the importance of good manners as she deals with the rude creatures of Wonderland. Alice maintains a superior attitude and behaves with solicitous indulgence toward those she believes are less privileged.
%
%----------------------------------------------------------------------------------------
\end{document} 
