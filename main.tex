\documentclass[letterpaper]{article}

\usepackage[sfdefault]{ClearSans}
\usepackage[T1]{fontenc}
\usepackage{tikz}
\usepackage{xcolor}
\usepackage{ragged2e}
\usepackage{hyperref}
\usepackage{pifont}
\usepackage{fontawesome5} % for icons

\usepackage[left=1cm,top=1.5cm,right=1cm,bottom=0.2cm,nohead,nofoot]{geometry}

% Color definitions
\definecolor{mainblue}{HTML}{0E5484}
\definecolor{maingray}{HTML}{0E5484}
\definecolor{gray}{HTML}{4D4D4D}

% Formatting
\pagestyle{empty}
\setlength{\parindent}{0pt}
\setlength{\tabcolsep}{0pt}

% Section headers
\newcounter{colorCounter}
\newcommand{\sectioncolor}[1]{%
  \colorbox{%
    \ifodd\value{colorCounter}mainblue\else maingray\fi%
  }{\textcolor{white}{\strut\hspace{6pt}#1\hspace{6pt}}}%
  \stepcounter{colorCounter}%
}
\renewcommand{\section}[1]{%
  \vspace{1.5em}
  {\Large\sectioncolor{#1}}\par
  \vspace{0.8em}
}
\renewcommand{\subsection}[1]{%
  \vspace{0.8em}
  {\large\color{gray} #1}\par
  \vspace{0.4em}
}

\newenvironment{cv}{%
  \begin{tabular*}{\textwidth}{@{\extracolsep{\fill}}ll}
}{%
  \end{tabular*}
}
\newcommand{\cvitem}[4]{%
  #1 & \parbox[t]{0.88\textwidth}{\textbf{#2} \hfill {\footnotesize#3}\\ #4 \vspace{\parsep}} \\
}

% Custom yellow star
\newcommand{\yellowstar}{\textcolor{yellow}{\ding{72}}}

\begin{document}

% Profile Header
\begin{minipage}[t]{0.7\linewidth}
  {\Huge\color{mainblue}Marco Maida}\\
  {\Large\color{black!80}Tech Lead, Inference Performance}\\
  \vspace{4pt}
  \faGlobe~\href{https://www.maida.me}{\texttt{maida.me}} \quad
  \faEnvelope~\href{mailto:mmmaidacs@gmail.com}{\texttt{mmmaidacs@gmail.com}}
\end{minipage}

\vspace{.5cm}

I’m a software engineer with over \textbf{ten years of experience} building systems that address complex, real-world problems. I worked in the fields of \textbf{industrial software}, \textbf{videogames}, \textbf{real-time systems},  \textbf{formal verification}, and \textbf{autonomous vehicles}.

I’m currently based in \textbf{London}, where I lead the inference performance team at \textbf{Wayve}. We focus on improving the latency and reliability of Wayve’s self-driving stack. I work mainly in \textbf{C++}, \textbf{Rust}, and \textbf{Python}, and I’m particularly interested in problems where \textbf{performance}, \textbf{correctness}, and \textbf{practical constraints} come together.

\vspace{.1cm}

\section{Professional Experience}

\begin{cv}
  \cvitem{Since 2022}{Software Engineer -- Autonomous Vehicles}{Wayve Ltd}{
    I designed, implemented, and maintained several key systems within Wayve's on-board software stack for autonomous vehicles. This includes sensor data collection, alignment, packaging, data upload, and real-time neural network inference. 
    Since December 2024, I have been serving as the \textbf{Tech Lead} of the inference performance team, responsible for the core inference component. \textit{(ROS2, Linux Kernel, C++, Rust, Python)}.
  }
  \cvitem{2016-2019}{Software Engineer -- Videogames}{34BigThings}{
    I contributed to five major game titles, focusing on game infrastructure, AI, gameplay, and development tools. I worked on both single-player and online multiplayer games shipped on Steam, PS4, XboxOne, Switch, and mobile platforms. \textit{(Unity3D, C\#, Unreal Engine, C++)}.
  }
  \cvitem{2015-2016}{Software Engineer -- Videogames, Simulation}{Maserati, Choralia}{
    I led two freelance B2B projects: developing an educational game for mobile and browsers, and collaborating to create a 3D visualization tool for product presentation. \textit{(Unity3D, C\#, JavaScript)}.
  }
  \cvitem{2013-2016}{Software Engineer -- Industrial Software}{R.O. srl}{
    I contributed to a suite of software solutions for glass processing factories, focusing on order tracking and optimizing machine work, product waste, and logistics. During my third year, I managed a team of 4 junior engineers. \textit{(C, C++, C\#, SQL)}.
  }
\end{cv}

\section{Education and Research}

\begin{cv}
  \cvitem{2022}{Research Internship}{Bloomberg LP}{
    I worked on accelerating SAT solving using GPUs. \textit{(C++, CUDA)}.
  }
  \cvitem{2019-2022}{PhD Student}{Max Planck Institute}{
    My research focused on formal verification and real-time systems, specifically verifying the timeliness of software systems. I published three papers and mentored three interns. \textit{(COQ, C, Rust)}.
  }
  \cvitem{2019-2022}{Master's in Computer Science}{Technische Universität Kaiserslautern}{
    I specialized in real-time systems as part of a joint master's and PhD program.
  }
  \cvitem{2016-2019}{Bachelor's in Computer Science}{Università degli Studi di Torino}{
    I specialized in computability and formal methods.
  }
\end{cv}

\subsection{Selected Publications}

\begin{cv}
  \cvitem{2025}{Claycode: Stylable and Deformable 2D Scannable Codes}{SIGGRAPH 2025}{
    \textit{\yellowstar \small  Journal publication in Transactions of Graphics. Selected for CAF trailer.}
    \\We introduced Claycodes, a new type of visual code that breaks free from traditional rigid grids like QR codes. Claycodes encode data as a tree structure and allow full customizability and deformation, with real-time scanning capabilities.	
    \scriptsize (\url{https://arxiv.org/abs/2505.08666})\normalfont.
  }
  \cvitem{2022}{From Intuition to Coq: A Case Study in Verified Response-Time Analysis, FIFO Scheduling}{RTSS 2022}{
    We developed a formally verified response-time analysis for FIFO schedulers, challenging traditional pen-and-paper methods. \scriptsize (\url{https://people.mpi-sws.org/~kbedarka/rtss22.pdf})\normalfont.
  }
  \cvitem{2021}{Foundational Response-Time Analysis as Explainable Evidence of Timeliness}{ECRTS 2022}{
    \textit{\yellowstar\small Outstanding Paper Award.}
    \\I developed POET, a tool for formally verified worst-case scenario timing analysis. \scriptsize (\url{https://drops.dagstuhl.de/opus/volltexte/2022/16336/pdf/LIPIcs-ECRTS-2022-19.pdf})\normalfont.
  }
\end{cv}

\vspace{100pt}
\section{Selected Open-Source Projects}

\begin{cv}
  \cvitem{2025}{Claycode}{\url{https://claycode.io}}{
    Inventor and lead contributor. Claycode includes a generator website, an Android scanner app, and a large amount of research code. 
  }
  \cvitem{2020-2022}{PROSA}{\url{https://gitlab.mpi-sws.org/RT-PROOFS/rt-proofs}}{
    Main contributor to PROSA, one of the leading formally-verified frameworks in the real-time systems community.
  }
  \cvitem{2022}{Treecode}{\url{www.maida.me/treecode}}{
    The precursor to Claycode, a novel 2D scannable code that encodes messages as unique trees.
  }
  \cvitem{2021}{POET}{\url{https://gitlab.mpi-sws.org/RT-PROOFS/POET}}{
    First-ever implementation of a foundational response-time analysis tool, created as part of my first academic publication.
  }
  \cvitem{2018}{Fast Mobile Cycle (FMC) Framework and Toolkit}{\url{www.github.com/34OpenThings}}{
    Developed a Unity3D framework that accelerates production-ready casual game creation, paired with a Python toolkit for bulk operations.
  }
\end{cv}

\end{document}
